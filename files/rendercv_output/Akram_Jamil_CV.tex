\documentclass[10pt, letterpaper]{article}

% Packages:
\usepackage[
    ignoreheadfoot, % set margins without considering header and footer
    top=1.25 cm, % seperation between body and page edge from the top
    bottom=1.25 cm, % seperation between body and page edge from the bottom
    left=1.25 cm, % seperation between body and page edge from the left
    right=1.25 cm, % seperation between body and page edge from the right
    footskip=0.625 cm, % seperation between body and footer
    % showframe % for debugging 
]{geometry} % for adjusting page geometry
\usepackage{titlesec} % for customizing section titles
\usepackage{tabularx} % for making tables with fixed width columns
\usepackage{array} % tabularx requires this
\usepackage[dvipsnames]{xcolor} % for coloring text
\definecolor{primaryColor}{RGB}{0, 79, 144} % define primary color
\usepackage{enumitem} % for customizing lists
\usepackage{fontawesome5} % for using icons
\usepackage{amsmath} % for math
\usepackage[
    pdftitle={Akram Jamil's CV},
    pdfauthor={Akram Jamil},
    pdfcreator={LaTeX with RenderCV},
    colorlinks=true,
    urlcolor=primaryColor
]{hyperref} % for links, metadata and bookmarks
\usepackage[pscoord]{eso-pic} % for floating text on the page
\usepackage{calc} % for calculating lengths
\usepackage{bookmark} % for bookmarks
\usepackage{lastpage} % for getting the total number of pages
\usepackage{changepage} % for one column entries (adjustwidth environment)
\usepackage{paracol} % for two and three column entries
\usepackage{ifthen} % for conditional statements
\usepackage{needspace} % for avoiding page brake right after the section title
\usepackage{iftex} % check if engine is pdflatex, xetex or luatex

% Ensure that generate pdf is machine readable/ATS parsable:
\ifPDFTeX
    \input{glyphtounicode}
    \pdfgentounicode=1
    % \usepackage[T1]{fontenc} % this breaks sb2nov
    \usepackage[utf8]{inputenc}
    \usepackage{lmodern}
\fi



% Some settings:
\AtBeginEnvironment{adjustwidth}{\partopsep0pt} % remove space before adjustwidth environment
\pagestyle{empty} % no header or footer
\setcounter{secnumdepth}{0} % no section numbering
\setlength{\parindent}{0pt} % no indentation
\setlength{\topskip}{0pt} % no top skip
\setlength{\columnsep}{0cm} % set column seperation
\pagenumbering{gobble} % no page numbering

\titleformat{\section}{\needspace{4\baselineskip}\bfseries\large}{}{0pt}{}[\vspace{1pt}\titlerule]

\titlespacing{\section}{
    % left space:
    -1pt
}{
    % top space:
    0.3 cm
}{
    % bottom space:
    0.2 cm
} % section title spacing

\renewcommand\labelitemi{$\circ$} % custom bullet points
\newenvironment{highlights}{
    \begin{itemize}[
        topsep=0.10 cm,
        parsep=0.10 cm,
        partopsep=0pt,
        itemsep=0pt,
        leftmargin=0.4 cm + 10pt
    ]
}{
    \end{itemize}
} % new environment for highlights

\newenvironment{highlightsforbulletentries}{
    \begin{itemize}[
        topsep=0.10 cm,
        parsep=0.10 cm,
        partopsep=0pt,
        itemsep=0pt,
        leftmargin=10pt
    ]
}{
    \end{itemize}
} % new environment for highlights for bullet entries


\newenvironment{onecolentry}{
    \begin{adjustwidth}{
        0.2 cm + 0.00001 cm
    }{
        0.2 cm + 0.00001 cm
    }
}{
    \end{adjustwidth}
} % new environment for one column entries

\newenvironment{twocolentry}[2][]{
    \onecolentry
    \def\secondColumn{#2}
    \setcolumnwidth{\fill, 4.5 cm}
    \begin{paracol}{2}
}{
    \switchcolumn \raggedleft \secondColumn
    \end{paracol}
    \endonecolentry
} % new environment for two column entries

\newenvironment{header}{
    \setlength{\topsep}{0pt}\par\kern\topsep\centering\linespread{1.5}
}{
    \par\kern\topsep
} % new environment for the header

\newcommand{\placelastupdatedtext}{% \placetextbox{<horizontal pos>}{<vertical pos>}{<stuff>}
  \AddToShipoutPictureFG*{% Add <stuff> to current page foreground
    \put(
        \LenToUnit{\paperwidth-1.25 cm-0.2 cm+0.05cm},
        \LenToUnit{\paperheight-0.625 cm}
    ){\vtop{{\null}\makebox[0pt][c]{
        \small\color{gray}\textit{Last updated in January 2025}\hspace{\widthof{Last updated in January 2025}}
    }}}%
  }%
}%

% save the original href command in a new command:
\let\hrefWithoutArrow\href

% new command for external links:
\renewcommand{\href}[2]{\hrefWithoutArrow{#1}{\ifthenelse{\equal{#2}{}}{ }{#2 }\raisebox{.15ex}{\footnotesize \faExternalLink*}}}


\begin{document}
    \newcommand{\AND}{\unskip
        \cleaders\copy\ANDbox\hskip\wd\ANDbox
        \ignorespaces
    }
    \newsavebox\ANDbox
    \sbox\ANDbox{}

    \begin{header}
        \textbf{\fontsize{24 pt}{24 pt}\selectfont Akram Jamil}

        \vspace{0.3 cm}

        \normalsize
        \mbox{\hrefWithoutArrow{mailto:akram.jamil@uwaterloo.ca}{\color{black}{\footnotesize\faEnvelope[regular]}\hspace*{0.13cm}akram.jamil@uwaterloo.ca}}%
        \kern 0.25 cm%
        \AND%
        \kern 0.25 cm%
        \mbox{\hrefWithoutArrow{tel:+1-647-406-5311}{\color{black}{\footnotesize\faPhone*}\hspace*{0.13cm}(647) 406-5311}}%
        \kern 0.25 cm%
        \AND%
        \kern 0.25 cm%
        \mbox{\hrefWithoutArrow{https://akramj.vercel.app/}{\color{black}{\footnotesize\faLink}\hspace*{0.13cm}akramj.vercel.app}}%
        \kern 0.25 cm%
        \AND%
        \kern 0.25 cm%
        \mbox{\hrefWithoutArrow{https://linkedin.com/in/akramjamil}{\color{black}{\footnotesize\faLinkedinIn}\hspace*{0.13cm}akramjamil}}%
        \kern 0.25 cm%
        \AND%
        \kern 0.25 cm%
        \mbox{\hrefWithoutArrow{https://github.com/akramj13}{\color{black}{\footnotesize\faGithub}\hspace*{0.13cm}akramj13}}%
    \end{header}

    \vspace{0.3 cm - 0.3 cm}


    \section{Education}



        
        \begin{twocolentry}{
            
            
        \textit{Sep 2024 to Apr 2029}}
            \textbf{University of Waterloo}

            \textit{BCFM in Computing \& Financial Management (Honors)}
        \end{twocolentry}

        \vspace{0.10 cm}
        \begin{onecolentry}
            \begin{highlights}
                \item \textbf{Majors:} Computer Science, Accounting \& Finance
                \item \textbf{Cumulative GPA}: 3.84/4.00
                \item \textbf{Relevant Coursework:} Analyzing Financial Markets in Python, Designing Programs, Algorithm Design and Data Structures, Techniques for Software Development, International Business
            \end{highlights}
        \end{onecolentry}



    
    \section{Experience}



        
        \begin{twocolentry}{
        \textit{Toronto, ON}    
            
        \textit{Jun 2022 to Aug 2022}}
            \textbf{Telecommunications Consultant Intern}
            
            \textit{Canada Cartage System Limited}
        \end{twocolentry}

        \vspace{0.10 cm}
        \begin{onecolentry}
            \begin{highlights}
                \item Managed server-side telecommunications systems (with over 5000+ devices), optimizing network performance and ensuring seamless communication across multiple departments.
                \item Analyzed and maintained telecommunications data to ensure compliance with industry standards and internal policies.
                \item Aided in the deployment of a company-wide project to upgrade the telecommunications infrastructure, reducing downtime by 20\%.
                \item Collaborated with IT teams to implement scalable communication solutions, supporting the company's logistics and transport operations.
            \end{highlights}
        \end{onecolentry}



    
    \section{Projects}



        
        \begin{twocolentry}{
            
            
        \textit{\href{https://github.com/sahilalamgir/AlzGuard}{GitHub Link}}}
            \textbf{AlzGuard – YIC (Youth Impact Challenge) Winning Project}
        \end{twocolentry}

        \vspace{0.10 cm}
        \begin{onecolentry}
            \begin{highlights}
                \item Engineered the front-end and AI model using React, Python, HTML, and CSS for an Alzheimer’s detection tool aimed at physicians, which won a \$1,000 prize at YIC.
                \item Aided in developing a convolutional neural network (CNN) to classify 2000+ images and qualitative clinical data to determine the likelihood of a patient having Alzheimer's Disease with 85.3\% accuracy.
                \item Collaborated in a team of 3 to integrate machine learning models (Random Forest, Meta Classifier) for analyzing MRI scans providing invaluable support in early diagnosis and patient care management.
            \end{highlights}
        \end{onecolentry}


        \vspace{0.2 cm}

        \begin{twocolentry}{
            
            
        \textit{\href{https://github.com/akramj13/ai-webscrape}{GitHub Link}}}
            \textbf{NLP Webscraping Tool}
        \end{twocolentry}

        \vspace{0.10 cm}
        \begin{onecolentry}
            \begin{highlights}
                \item Built an AI-Powered Webscraper that can pull data from any website, given a prompt, using a Meta's Ollama AI (Llama ver. 3.2).
                \item Created an NLP-powered solution that dynamically extracted relevant content from website URLs based on user prompts, achieving accurate data retrieval across 100+ test cases, enhancing web-scraping efficiency.
                \item Designed front-end in React.js Framework and the Selenium Python Package and connected using a Flask server.
            \end{highlights}
        \end{onecolentry}


        \vspace{0.2 cm}

        \begin{twocolentry}{
            
            
        \textit{\href{https://github.com/akramj13/ai-stock-predictor}{GitHub Link}}}
            \textbf{Random Forest Classifier for Stock Predictions}
        \end{twocolentry}

        \vspace{0.10 cm}
        \begin{onecolentry}
            \begin{highlights}
                \item Developed a custom predictive analytics tool to forecast stock price movements based on historical financial data over the past 10 years.
                \item Programmed a Random Forest Classifier in Python using Scikit-Learn, trained on 10,000+ data points of stock prices and trading volumes, combined with technical indicators to achieve a 15\% improvement in predictive accuracy for market trends.
                \item Data is taken from Yahoo Finance using the yfinance library in Python and recommends a stock that has a greater than 55\% chance of rising in value.
            \end{highlights}
        \end{onecolentry}



    
    \section{Technologies}



        
        \begin{onecolentry}
            \textbf{Languages:} C/C++, Python, Dart, SQL, Java, JavaScript, TypeScript, HTML, CSS, Scheme, Racket
        \end{onecolentry}

        \vspace{0.2 cm}

        \begin{onecolentry}
            \textbf{Libraries/Frameworks:} Matplotlib, NumPy, NumPy Financials, pandas, yfinance, Flask, Selenium, BeautifulSoup, React, Next.js, TailwindCSS, Flutter
        \end{onecolentry}

        \vspace{0.2 cm}

        \begin{onecolentry}
            \textbf{Tools/Technologies:} Visual Studio Code, Android Studio, Jupyter Notebooks, LaTeX, Blender, Git/GitHub, Figma
        \end{onecolentry}


    

\end{document}